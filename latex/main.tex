% TEMPLATE for Usenix papers, specifically to meet requirements of
%  USENIX '05
% originally a template for producing IEEE-format articles using LaTeX.
%   written by Matthew Ward, CS Department, Worcester Polytechnic Institute.
% adapted by David Beazley for his excellent SWIG paper in Proceedings,
%   Tcl 96
% turned into a smartass generic template by De Clarke, with thanks to
%   both the above pioneers
% use at your own risk. Complaints to /dev/null.
% make it two column with no page numbering, default is 10 point

% Munged by Fred Douglis <douglis@research.att.com> 10/97 to separate
% the .sty file from the LaTeX source template, so that people can
% more easily include the .sty file into an existing document. Also
% changed to more closely follow the style guidelines as represented
% by the Word sample file.

% Note that since 2010, USENIX does not require endnotes. If you want
% foot of page notes, don't include the endnotes package in the 
% usepackage command, below.

% This version uses the latex2e styles, not the very ancient 2.09 stuff.
\documentclass[letterpaper,twocolumn,10pt]{article}
\usepackage{usenix,epsfig,endnotes}
\begin{document}

%don't want date printed
\date{}

%make title bold and 14 pt font (Latex default is non-bold, 16 pt)
\title{\Large \bf Hycrypt: XTS Full Disk Encryption Alternative with Chacha and Poly1305}

%for single author (just remove % characters)
\author{
{\rm Bernard Dickens}\\
University of Chicago\\
bd3@uchicago.edu
\and
{\rm Ariel Feldman}\\
University of Chicago\\
arielfeldman@uchicago.edu
\and
{\rm Henry Hoffmann}\\
University of Chicago\\
hankhoffmann@uchicago.edu
% copy the following lines to add more authors
% \and
% {\rm Name}\\
%Name Institution
} % end author

\maketitle

% Use the following at camera-ready time to suppress page numbers.
% Comment it out when you first submit the paper for review.
\thispagestyle{empty}

\subsection*{Abstract}
Block-level full-disk encryption (FDE) is a non-trivial engineering problem rife with competing interests and
messy tradeoffs. An ideal FDE solution must ensure data confidentiality, data integrity, and efficient operation
within a potentially highly-constrained total energy budget, such as with a mobile or embedded system. All of this must
be accomplished without adding significant overhead to system performance. The current de-facto FDE standard,
AES in XTS mode, achieves only data confidentiality and does so by sacrificing energy and system performance while
abandoning data integrity guarantees entirely. However, recent work on faster and more energy-efficient alternatives to
the AES cipher allow us to revisit some of the design decisions that underpin AES-XTS. Specifically, we propose Hycat: an
FDE alternative to AES-XTS utilizing the Chacha20 stream cipher and Poly1305 MAC algorithm.

\section{Introduction}
Somewhat concise introduction like in the Chacha paper. Is this worth it? Briefly address project meaningfulness.
Status quo disk encryption has a non-trivial cost. Enumerate any potential tradeoffs. Describe extra benefits of a
Chacha-LFS construction (i.e. integrity checking, simpler design) over XTS.

Caveats, major limitations, and how they're handled.

\begin{itemize}
  \item Here we summarize the main contributions (as with the brief introduction) with a focus on the justification for
  the project's existence. Tease apart and enumerate any other contributions at the end (see MEANTIME paper for format).
  \item Argue that this is a meaningful project by showing (via chart) that encryption has a non-trivial cost. We would 
  use the initial FDE vs  NFDE on RD experiment charts.
  \item Do we address limitations here or in the conclusion at the end?
  \item Argue that making a stream cipher work for FDE in this instance is non-trivial and comes with costs. Characterize.
  ``We save X at the cost of Y''. DE is the problem. The solution is Z. It costs Y. If willing to pay.
  \item Need to tell what the above costs are, if and how they can be minimized, and what trade-offs they constitute.
\end{itemize}

\section{Motivational Example}
(mobile phone battery life example, android K/L/M FDE stats)

\section{Related Works}
(broken off from the introduction)

\subsection{AES-XTS}
AES-XTS background information

\subsection{Salsa20/Chacha20}
Stream Cipher and Salsa20/Chacha20 background information

Show Chacha as a stream cipher to be faster than AES in a stream cipher-ey mode (CTR or GSM to compare with integrity
checking). Need to show that these modes of AES are always? faster than the very slow double-keyed XTS construction.
This would establish that there is some slack to be played with.

\section{Hycrypt Full Disk Encryption}
(maybe a short blurb here; see MEANTIME paper for direction)

\subsection{Threat Model}
(this will be very similar to the threat model established by the XTS project and writeups)

\subsection{Design and Implementation}
Describe LFS construction in detail, explain design decisions, argue security and preservation of crypto guarantees via
assumed cryptographic primitives, describe perceived extra benefits (i.e. integrity checking, simpler design) over XTS

\section{Experimental Setup}
\begin{itemize}
  \item RD+Fuse-Ext4 is the baseline for all experiments
  \item Crafting the story by gathering meaningful data using the performance of so-called ``I/O bound applications'' as
  metrics. Currently under consideration: git, mobile bench (as a suite), Chrome/FF (mobile and non-mobile; talk to
  Connor about FF), VLC x264 playback, graphg/social graph, Ferret (read paper), SQLite/Access/No-SQL local DB,
  Filezilla FTP, Google Maps offline map caching, mobile SMS history
  \item The above would be evaluated across NFDE (RDFx4), dmcrypt (RDFx4D), then LFS-Chacha (RDLC) and LFS-AESGCM (RDLA)
  [and also RDLFS for good measure]
  \item Describe how I/O bound applications were classified as such and why they were selected (explain I/O profiling
  tool usage and setup)
  \item Any official standard benchmarking suites?
\end{itemize}

\section{Experimental Evaluation}
Describe evaluation of Chacha-LFS versus AES-XTS
\begin{itemize}
  \item Charts showing actual benchmarks of i/o bound and/or i/o heavy applications evaluated across testbed of filesystems
  \item Show a chart depicting energy use on mobile or device with typical pure NAND flash + dmcrypt FDE versus NAND
  flash + LFS Chacha versus NAND flash + LFS Chacha in ``energy saving mode'' (saving garbage collection etc for later).
  We would expect to see a decrease in energy use from dmcrypt down to LFS in energy saving mode but little to no
  degredation i performance from dmcrypt to LFS! 
  \item Chart showing how the LFS performs with Chacha20+Poly1305 as stream cipher versus AES-GCM as stream cipher. This
  would demonstrated that the performance gains are not just from using the LFS over Ext4. We should expect to see the LFS
  under C20+P1305 outperform LFS under AES-GCM.
\end{itemize}

\section{Conclusion}
\begin{itemize}
  \item Address (defend?) the use of RD+Fuse-Ext4 as a baseline. It would be best if we could relate any RD+Fuse-Ext4 based
  performance metrics with performance on a standard SSD (in Experimental Evaluation) so that it does not look like we're
  trying to hide anything.
  \item Address limitation: why would this project remain important even if SSDs sped up dramatically? (b/c energy savings)
  \item Address limitation: would this project remain relevant if hardware accelerated AES become prevalent? (yes b/c
  embedded devices and energy savings)
  \item Address choice of benchmarking suite
\end{itemize}

\section{Acknowledgments}

A polite author such as myself always includes acknowledgments. Thank everyone, especially those who funded the work.

\section{Availability}

Release LFS code, LFS+Chacha20, LFS+AESGCM, I/O profiling tool(s)/setup, etc. Do we released entire repo/all experimental
code as well?

\begin{center}
{\tt ftp.site.dom/pub/myname/Wonderful}\\
\end{center}

Will also throw up some sort of homepage or webpage at some point:

\begin{center}
{\tt http://www.site.dom/\~{}myname/SWIG}
\end{center}

{\footnotesize \bibliographystyle{acm} \bibliography{references.bib}}

% \theendnotes

\end{document}
