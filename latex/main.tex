% TEMPLATE for Usenix papers, specifically to meet requirements of
%  USENIX '05
% originally a template for producing IEEE-format articles using LaTeX.
%   written by Matthew Ward, CS Department, Worcester Polytechnic Institute.
% adapted by David Beazley for his excellent SWIG paper in Proceedings,
%   Tcl 96
% turned into a smartass generic template by De Clarke, with thanks to
%   both the above pioneers
% use at your own risk. Complaints to /dev/null.
% make it two column with no page numbering, default is 10 point

% Munged by Fred Douglis <douglis@research.att.com> 10/97 to separate
% the .sty file from the LaTeX source template, so that people can
% more easily include the .sty file into an existing document. Also
% changed to more closely follow the style guidelines as represented
% by the Word sample file.

% Note that since 2010, USENIX does not require endnotes. If you want
% foot of page notes, don't include the endnotes package in the 
% usepackage command, below.

% This version uses the latex2e styles, not the very ancient 2.09 stuff.
\documentclass[letterpaper,twocolumn,10pt]{article}
\usepackage{usenix,epsfig,endnotes}
\begin{document}

%don't want date printed
\date{}

%make title bold and 14 pt font (Latex default is non-bold, 16 pt)
\title{\Large \bf Hycrypt: More Performant Full Disk Encryption with Chacha and Poly1305}

%for single author (just remove % characters)
\author{
{\rm Bernard Dickens}\\
University of Chicago
bd3@cs.uchicago.edu
\and
{\rm Ariel Feldman}\\
University of Chicago
arielfeldman@cs.uchicago.edu
\and
{\rm Henry Hoffmann}\\
University of Chicago
hankhoffmann@cs.uchicago.edu
% copy the following lines to add more authors
% \and
% {\rm Name}\\
%Name Institution
} % end author

\maketitle

% Use the following at camera-ready time to suppress page numbers.
% Comment it out when you first submit the paper for review.
\thispagestyle{empty}


\subsection*{Abstract}
Have Your Cake and eat it too \\
Summarize problem, introduction/contributions, implementation, conclusion in 250? interesting words or less.

\section{Introduction}
Somewhat concise introduction like in the Chacha paper. Is this worth it? Briefly address project meaningfulness.
Status quo disk encryption has a non-trivial cost. Enumerate any potential tradeoffs. Describe extra benefits of a
Chacha-LFS construction (i.e. integrity checking, simpler design) over XTS.

Caveats, major limitations, and how they're handled.

\begin{itemize}
  \item Here we summarize the main contributions (as with the brief introduction) with a focus on the justification for
  the project's existence. Tease apart and enumerate any other contributions.
  \item Argue that this is a meaningful project by showing that encryption has a non-trivial cost [use initial FDE vs
  NFDE experiments]
  \item Chart FDE vs NFDE for reads/writes of random data (perhaps dd tests, perhaps random data file tests) assists
  \item Limitation: hardware accelerated AES makes the disparity between NFDE and FDE disappear; countered by HAAES not being
  available very widly on many mobile devices, especially embedded
  \item Show Chacha as a stream cipher to be faster than AES in a stream cipher-ey mode (CTR or GSM to compare
  with integrity checking). Need to show that these modes of AES are always? faster than the very slow double-keyed XTS
  construction. This would establish that there is some slack to be played with between AES-XTS and other modes and something like a chacha.
  \item Argue (perhaps in the Salsa20 section below) that making a stream cipher work for FDE in this instance is non-trivial
  and comes with costs. Characterize. ``We save X at the cost of Y''. DE is the problem. The solution is Z. It costs Y. If willing to pay.
  \item Need to tell what these costs are, if and how they can be minimized, and what trade-offs they constitute.
\end{itemize}


\section{Related Works}
(broken off from the introduction)
\subsection{AES-XTS}
\subsection{Salsa20/Chacha20}

\section{Hycrypt Full Disk Encryption}
\subsection{Threat Model}
(this will be very similar to the threat model established by the XTS project and writeups)

\subsection{Design and Implementation}
Describe LFS construction in detail, explain design decisions, preservation of security guarantees, extra benefits
(i.e. integrity checking, simpler design) over XTS

\section{Experimental Setup}
Describe evaluation of Chacha-LFS versus AES-XTS

\section{Experimental Evaluation}
\begin{itemize}
  \item Charts showing actual benchmarks of i/o bound and/or i/o heavy applications (i.e. git, several others) under
  NFDE (control), AES-XTS, and ChaCha-LFS
  \item Benchmarking suites? TBD
\end{itemize}

\section{Conclusion}


\section{Acknowledgments}

A polite author always includes acknowledgments. Thank everyone, especially those who funded the work. 

\section{Availability}

It's great when this section says that MyWonderfulApp is free software, available via anonymous FTP from

\begin{center}
{\tt ftp.site.dom/pub/myname/Wonderful}\\
\end{center}

Also, it's even greater when you can write that information is also available on the Wonderful homepage at 

\begin{center}
{\tt http://www.site.dom/\~{}myname/SWIG}
\end{center}

{\footnotesize \bibliographystyle{acm} \bibliography{references.bib}}

% \theendnotes

\end{document}
